\documentclass[12pt,a4paper,ngerman]{scrartcl}

\usepackage{babel}
\usepackage{todonotes}

\author{77.~Konferenz der deutschsprachigen
  Mathematikfachschaften}
\title{Resolution gegen Fremdenfeindlichkeit}

\begin{document}
\maketitle{}

Im Zuge der Flüchtlingskrise\todo{Besseres Wort einsetzen.} finden
fremdenfeindliche Bewegungen und Pegida\todo{Entscheiden, ob wir
  allgemein „fremdenfeindlich“ oder konkrete Namen nennen möchten.}
zunehmend Zuspruch. Ebenso ist ein Anstieg politisch motivierter
Straftaten zu verzeichnen. Entstehende Ängste und Unsicherheiten
führen zu Spannungen in der Gesellschaft.

Wir, die 77.~Konferenz der deutschsprachigen Mathematikfachschaften,
möchten uns von diesen Bewegungen distanzieren und verurteilen
Fremdenfeindlichkeit\todo{Anderes Wort?} zutiefst. Wir bitten die
Hochschulen, sich für Internationalität und Austausch in Forschung,
Lehre und Gesellschaft auszusprechen und sich daher ebenso zu
positionieren. Dazu gehört insbesondere, im Rahmen ihrer Möglichkeiten
zu Bildung und Integration für Geflohene beizutragen, sowie den Zugang
zum Studium möglichst\todo{Synonym?} einfach zu gestalten. Wir
erwarten einen sachlichen und vorurteilsfreien Diskurs zu politischen
und wissenschaftlichen Themen, um dem bewussten\todo{Können wir das
  belegen?} oder billigend in Kauf genommen Schüren von Ängsten
entgegenzutreten.

Diese Positionen ergeben sich daraus, dass wir eine tolerante und
weltoffene Gesellschaft vertreten, die in klarem Widerspruch zu
fremdenfeindlichem Gedankengut steht.

\end{document}

%%% Local Variables:
%%% mode: latex
%%% TeX-master: t
%%% End:
