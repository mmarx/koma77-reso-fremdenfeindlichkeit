\documentclass[12pt,a4paper,ngerman,DIV=14,draft]{scrartcl}

\usepackage[utf8]{inputenc}
\usepackage{babel}
\usepackage{url}
\usepackage[obeyFinal]{todonotes}

\renewcommand{\thefootnote}{\fnsymbol{footnote}}
\author{77.~Konferenz der deutschsprachigen Mathematikfachschaften}
\title{Resolution gegen Fremdenfeindlichkeit}

\begin{document}
\maketitle{}

\pagenumbering{gobble}

Im Zuge der Flüchtlingskrise finden fremdenfeindliche Bewegungen
zunehmend Zuspruch. Ebenso ist ein
Anstieg\footnote{\url{https://www.bmi.bund.de/SharedDocs/Pressemitteilungen/DE/2015/05/pks-und-pmk-2014.html}}
politisch motivierter Straftaten zu verzeichnen. Wachsende Ängste und
Unsicherheiten führen zu Spannungen in der Gesellschaft.

Wir, die 77.~Konferenz der deutschsprachigen Mathematikfachschaften,
lehnen diese Bewegungen ab und verurteilen Fremdenfeindlichkeit
zutiefst. Wir bitten die Hochschulen, sich für Internationalität und
Austausch in Forschung, Lehre und Gesellschaft auszusprechen und sich
damit ebenso zu positionieren.

Dazu gehört insbesondere, im Rahmen ihrer Möglichkeiten zu Bildung und
Integration für Geflohene beizutragen sowie den Zugang zum Studium
möglichst einfach zu gestalten. Wir erwarten einen sachlichen und
vorurteilsfreien Diskurs zu politischen und wissenschaftlichen Themen,
um dem
bewussten\footnote{\url{https://www.mimikama.at/?s=pegida}}\footnote{\url{http://www.dnn.de/Mitteldeutschland/Polizeiticker-Mitteldeutschland/Falschmeldungen-bei-Facebook-Sachsens-Polizei-geht-in-die-Offensive}}
oder billigend in Kauf genommen Schüren von Ängsten entgegenzutreten.

Diese Positionen ergeben sich daraus, dass wir fremdenfeindliches
Gedankengut in klarem Widerspruch zum von uns vertretenen Bild einer
toleranten und weltoffenen Gesellschaft sehen. Ein globaler Austausch
kommt Forschung, Lehre und Charakterbildung zu Gute. Wir begrüssen die
Auseinandersetzung mit konträren Meinungen. Eine pauschale Ablehnung
oder gar Diskreditierung fundierter Gegenpositionen, etwa durch die
Bezeichnung als „Lügenpresse“, läuft den wissenschaftlichen Arbeits-
und Diskussionsprinzipien zuwider.

Zusammenfassend sehen wir im Umgang mit Geflohenen und Zugewanderten
aus der ganzen Welt keine Gefahr, sondern im Gegenteil eine
Chance. Wir wünschen uns, dass insbesondere die hier Angeschriebenen
diese Chance ergreifen.

\end{document}

%%% Local Variables:
%%% mode: latex
%%% TeX-master: t
%%% ispell-local-dictionary: de_DE
%%% End:
